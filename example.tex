\documentclass{article}
\begin{document}
Assume that \(\{x_n\}_{n=1}^\infty \subset \mathbb{R}^2\) satisfies \(\|x_n\|_2 \leq C \) for all \( n \geq 1,\) so that every \(x_n\) lies in the closed disk
\[
D = \{x \in \mathbb{R}^2 : \|x\|_2 \leq C\}.
\]
Enclose the disk \(D\) in a closed square \(S\), which is compact. Divide \(S\) into four congruent closed squares. Since there are infinitely many points \(\{x_n\}\) in \(S\) and finitely many subsquares, by the Pigeonhole Principle at least one of these subsquares, say \(S_1\), contains infinitely many \(x_n\)'s. We repeat the process on \(S_1\): subdivide \(S_1\) into four congruent closed subsquares. At least one of these subsquares, denote it by \(S_2\), must contain infinitely many points of \(\{x_n\}\). Continue this process to obtain a nested sequence of closed squares:
\[
S \supset S_1 \supset S_2 \supset S_3 \supset \cdots,
\]
where for each \(k\) the square \(S_k\) contains infinitely many points of \(\{x_n\}\), and the diameter of \(S_k\) tends to zero as \(k \to \infty\). 
\end{document} 